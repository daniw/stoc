% coding:utf-8

% Ausführen in R: 
% Sweave("C:/Daten/Daniel/studium/git_repo/sem2/stoc/sw10/sw10_3.Rnw",encoding='UTF-8')

\section{Aufgabe 3}

\subsection{a}
Es handelt sich um einen gepaarten Test, weil jeder Athlet mit beiden Schuhen 
getestet wird. 

\subsection{b}
\begin{enumerate}
  \item Modell \\
    $ D \sim \mathcal{N}(\mu,\hat{\sigma}_{x-y})$
  \item Nullhypothese \\
    $H_0: \mu=\mu_0=0$\\
    Alternativhypothese: \\
    $H_A: \mu<\mu_0$
  \item Teststatistik \\
    $T = \frac{\sqrt{n}\cdot(\bar{x}-\bar{y}-\mu_0)}{\hat{\sigma}_{x-y}}$ \\
    $T = \frac{\sqrt{10}\cdot(-0.22-0)}{0.26} = -2.676$ \\
    Verteilung der Teststatistik unter $H_0$\\
    $T \sim t_{n-1}$
  \item Signifikanzniveau \\
    $\alpha=0.05$
  \item Verwerfungsbereich \\
    $K = (-\infty,-t_{n-1,1-\alpha}]$
\begin{Schunk}
\begin{Sinput}
> qt(p=0.05,df=10)
\end{Sinput}
\begin{Soutput}
[1] -1.812461
\end{Soutput}
\end{Schunk}
    $K = (-\infty,-1.812461]$
  \item Testentscheid \\
    $T$ liegt im Verwerfungsbereich. 
\end{enumerate}

\subsection{c}
