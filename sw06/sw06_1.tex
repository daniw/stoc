% coding:utf-8

% Ausführen in R: 
% Sweave("C:/Daten/Daniel/studium/git_repo/sem2/stoc/sw06/sw06_1.Rnw",encoding='UTF-8')

\section{Aufgabe 1}

\subsection{a}
\begin{enumerate}
  \item Modell \\
        $X =$ \#Patienten, die auf die Behandlung ansprechen\\
        $X \sim Bin(n,\pi)$\\
        $X \sim Bin(16,\pi)$
  \item Nullhypothese \\
        $H_0: \pi_0 = 0.15$\\
        $H_A: \pi > \pi_0 = 0.15$
  \item Teststatistik \\
        $T =$ \#Patienten, die auf die Behandlung ansprechen \\
        $T \sim Bin(16,\pi_0 = 0.15)$
  \item Signifikanzniveau \\
        $\alpha = 0.05$
  \item Verwerfungsbereich \\
        $K = (c,n), \quad P_{H_0}(T \in (c,n)) \stackrel{\sim}{\leq} \alpha$\\
        $P_{H_0}(T \geq c) = 1 - P(T \leq \underbrace{c - 1}_{= 5})$\\
        $\Rightarrow c - 1 = 5 \quad c = 6$\\
        $K = [6, 16]$
  \item Testentscheid\\
        $t = 5$
\end{enumerate}

\subsection{b}
5

\subsection{c}
$P_{\pi=0.3}(T \in K)$\\
$ = P_{\pi=0.3}(T \geq 6) = 1 - P_{\pi=0.3}(T \leq 5) 
= 1 - \sum_{K=0}^{5}\left(\begin{array}{l}16\\K\end{array}\right) 
~ 0.3^K ~ 0.7^{16-K}$
\begin{Schunk}
\begin{Sinput}
> 1 - pbinom(5,size=16,prob=0.3)
\end{Sinput}
\begin{Soutput}
[1] 0.3402177
\end{Soutput}
\end{Schunk}
Oder
\begin{Schunk}
\begin{Sinput}
> sum(dbinom(6:16,size=16,prob=0.3))
\end{Sinput}
\begin{Soutput}
[1] 0.3402177
\end{Soutput}
\end{Schunk}

