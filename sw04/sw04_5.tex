% coding:utf-8

\section{Aufgabe 5}
\subsection{a}
<<>>=
dbinom(10,size=50,prob=0.2)
@

\subsection{b}
<<>>=
sum(dbinom(0:5,size=50,prob=0.2))
@

\subsection{c}
<<>>=
sum(dbinom(15:50,size=50,prob=0.2))
@

\subsection{d}
$c$ für $P(X \leq c) \approx 0.99 wird von Hand empirisch gefunden. 
<<>>=
sum(dbinom(0:20,size=50,prob=0.2))
sum(dbinom(0:15,size=50,prob=0.2))
sum(dbinom(0:16,size=50,prob=0.2))
sum(dbinom(0:17,size=50,prob=0.2))
@
Das ist in R sicher auch automatisch ohne For-Schleife möglich. 

\subsection{e}
<<>>=

@

\subsection{f}
<<>>=

@

\subsection{g}
<<>>=

@
