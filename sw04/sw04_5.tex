% coding:utf-8

\section{Aufgabe 5}
\subsection{a}
\begin{Schunk}
\begin{Sinput}
> dbinom(10,size=50,prob=0.2)
\end{Sinput}
\begin{Soutput}
[1] 0.139819
\end{Soutput}
\end{Schunk}

\subsection{b}
\begin{Schunk}
\begin{Sinput}
> sum(dbinom(0:5,size=50,prob=0.2))
\end{Sinput}
\begin{Soutput}
[1] 0.04802722
\end{Soutput}
\end{Schunk}

\subsection{c}
\begin{Schunk}
\begin{Sinput}
> sum(dbinom(15:50,size=50,prob=0.2))
\end{Sinput}
\begin{Soutput}
[1] 0.06072208
\end{Soutput}
\end{Schunk}

\subsection{d}
$c$ für $P(X \leq c) \approx 0.99$ wird von Hand empirisch gefunden. 
\begin{Schunk}
\begin{Sinput}
> sum(dbinom(0:20,size=50,prob=0.2))
\end{Sinput}
\begin{Soutput}
[1] 0.9996793
\end{Soutput}
\begin{Sinput}
> sum(dbinom(0:15,size=50,prob=0.2))
\end{Sinput}
\begin{Soutput}
[1] 0.9691966
\end{Soutput}
\begin{Sinput}
> sum(dbinom(0:16,size=50,prob=0.2))
\end{Sinput}
\begin{Soutput}
[1] 0.9855583
\end{Soutput}
\begin{Sinput}
> sum(dbinom(0:17,size=50,prob=0.2))
\end{Sinput}
\begin{Soutput}
[1] 0.9937392
\end{Soutput}
\end{Schunk}
Das ist in R sicher auch automatisch ohne For-Schleife möglich. 

\subsection{e}
\begin{Schunk}
\begin{Sinput}
> plot(0:1000,dpois(200,0:1000),type='l')
> dpois(200,200)
\end{Sinput}
\begin{Soutput}
[1] 0.02819773
\end{Soutput}
\end{Schunk}

\subsection{f}
\begin{Schunk}
\begin{Sinput}
> sum(dpois(200,0:200))
\end{Sinput}
\begin{Soutput}
[1] 0.4953046
\end{Soutput}
\end{Schunk}

\subsection{g}
\begin{Schunk}
\begin{Sinput}
> sum(dpois(200,190:210))
\end{Sinput}
\begin{Soutput}
[1] 0.5420267
\end{Soutput}
\end{Schunk}
