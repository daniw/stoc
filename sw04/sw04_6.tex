% coding:utf-8

% Ausführen in R: 
% Sweave("C:/Daten/Daniel/studium/git_repo/sem2/stoc/sw04/sw04_6.Rnw",encoding='UTF-8')

\section{Aufgabe 6}
\subsection{a}
\[ P(X=6) = \left(\begin{array}{l}6\\6\end{array}\right) 
\cdot \left(\frac{1}{49}\right)^6 \cdot \left(1 - \frac{1}{49}\right)^{6-6} 
= \frac{6!}{6! \cdot (6-6)!} \cdot \left(\frac{1}{49}\right)^6 
\cdot \left(1 - \frac{1}{49}\right)^0 = 7.22 \cdot 10^{-11} \]
R:
\begin{Schunk}
\begin{Sinput}
> dbinom(6,size=6,prob=1/49)
\end{Sinput}
\begin{Soutput}
[1] 7.224762e-11
\end{Soutput}
\end{Schunk}

\subsection{b}
Anzahl Zahlen mit 7 Ziffern 1: 
\[ \left(\begin{array}{l}15\\7\end{array}\right) 
= \frac{15!}{7! \cdot (15-7)!} = 6435 \]
Zahlen insgesamt: 
\[ 2^{15} = 32768 \]
