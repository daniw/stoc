% coding:utf-8

\section{Aufgabe 2}
\begin{Schunk}
\begin{Sinput}
> dist=c(100,200,400,800,1000,1500,2000,3000,5000,10000,20000,25000,30000)
> time=c(9.9,19.8,43.8,103.7,136,213.1,296.2,457.6,793.0,1650.8,3464.4,4495.6,5490.4)
> # summary(lm(time ~ dist))
\end{Sinput}
\end{Schunk}

\subsection{a}
Es gibt einen signifikanten Zusammenhang zwischen Distanz und Zeit, 
da Pr(>|t|) < 2e-16 ist. 

\subsection{b}
\begin{Schunk}
\begin{Sinput}
> 0.18170-0.00173*qt(p=0.975,df=11)
\end{Sinput}
\begin{Soutput}
[1] 0.1778923
\end{Soutput}
\begin{Sinput}
> 0.18170+0.00173*qt(p=0.975,df=11)
\end{Sinput}
\begin{Soutput}
[1] 0.1855077
\end{Soutput}
\end{Schunk}
ii

\subsection{c}
Mittels Formel: 
\begin{Schunk}
\begin{Sinput}
> time[5]-(0.18170*dist[5]+(-62.59296))
\end{Sinput}
\begin{Soutput}
[1] 16.89296
\end{Soutput}
\end{Schunk}
Mit R-Funktionen
\begin{Schunk}
\begin{Sinput}
> residuals(lm(time~dist))[5]
\end{Sinput}
\begin{Soutput}
      5 
16.8959 
\end{Soutput}
\end{Schunk}

\subsection{d}
Nein

\subsection{e}
\begin{Schunk}
\begin{Sinput}
> sd(residuals(lm(time~dist)))
\end{Sinput}
\begin{Soutput}
[1] 60.01135
\end{Soutput}
\end{Schunk}
Das Modell ist nur für grosse Werte von \verb?time? brauchbar. 

\subsection{f}
Wahrscheinlich wirkt zusätzlich ein quadratisch wirkender Teil mit. 

\subsection{g}
$Zeit_i 
= \beta_0 + \beta_1 \cdot Distanz + \beta_2 \cdot Distanz^2 + \epsilon_i$
