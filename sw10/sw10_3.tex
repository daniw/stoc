% coding:utf-8

% Ausführen in R: 
% Sweave("C:/Daten/Daniel/studium/git_repo/sem2/stoc/sw10/sw10_3.Rnw",encoding='UTF-8')

\section{Aufgabe 3}

\subsection{a}
\begin{enumerate}
  \item Modell \\
        $ X \sim N(\mu, {\sigma_x}^2), \sigma_x \text{bekannt} $
  \item Nullhypothese \\
        $ H_0: \mu = \mu_0 $ \\
        $ H_A: \mu > \mu_0 $
  \item Teststatistik \\
        \[ Z = \frac{\sqrt{n}\cdot (\bar{X_n} - \mu_0)}{\sigma_x} \] 
        Verteilung der Teststatistik unter $ H_0: Z \sim N(0,1) $
  \item Signifikanzniveau \\
        $ \alpha = 0.05 $
  \item Verwerfungsbereich der Teststatistik \\
        $ K = [\Phi^{-1}(1-\alpha), \infty] $
\begin{Schunk}
\begin{Sinput}
>         qnorm(p=0.95)
\end{Sinput}
\begin{Soutput}
[1] 1.644854
\end{Soutput}
\end{Schunk}
  \item Testentscheid \\
        $ Z = \dfrac{\sqrt{16} \cdot (204.2 - 200)}{10} = 1.68 $ \\
        $Z$ liegt im Verwerfungsbereich $\rightarrow$ $H_0$ kann verworfen 
        werden. 
\end{enumerate}

\subsection{b}
\begin{Schunk}
\begin{Sinput}
> 1-pnorm(q=205,mean=205,sd=10)
\end{Sinput}
\begin{Soutput}
[1] 0.5
\end{Soutput}
\end{Schunk}

\subsection{c}
\begin{Schunk}
\begin{Sinput}
> pnorm(q=205,mean=200,sd=10)
\end{Sinput}
\begin{Soutput}
[1] 0.6914625
\end{Soutput}
\end{Schunk}

\subsection{d}
\begin{enumerate}
  \item Modell \\
        $ X \sim N(\mu, {\sigma_x}^2), \sigma_x 
        \text{wird durch $\hat{\sigma_x}$}$ geschätzt. 
  \item Nullhypothese \\
        $ H_0: \mu = \mu_0 $ \\
        $ H_A: \mu > \mu_0 $
  \item Teststatistik \\
        \[ T = \frac{\sqrt{n}\cdot (\bar{X_n} - \mu_0)}{\hat{\sigma_x}} \] 
        Verteilung der Teststatistik unter $ H_0: T \sim t_{15} $
  \item Signifikanzniveau \\
        $ \alpha = 0.05 $
  \item Verwerfungsbereich der Teststatistik \\
        $ K = [t_{n-1;1-\alpha}, \infty] $
\begin{Schunk}
\begin{Sinput}
>         qt(p=0.95,df=16-1)
\end{Sinput}
\begin{Soutput}
[1] 1.75305
\end{Soutput}
\end{Schunk}
  \item Testentscheid \\
        $ T = \dfrac{\sqrt{16} \cdot (204.2 - 200)}{10} = 1.68 $ \\
        $T$ liegt im Verwerfungsbereich $\rightarrow$ $H_0$ kann verworfen 
        werden. 
\end{enumerate}
